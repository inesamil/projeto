%
% Análise
%
\section*{Análise} \label{Analise}

Para a realização deste projeto encontrámos algumas dificuldades que se prendem com o facto de os produtos não terem rótulos digitais. Isto é um problema para a concretização do projeto, na medida em que se torna menos eficiente a obtenção dos dados presentes nos produtos. Contudo, assumindo que este dilema é resolvido fora do âmbito do projeto, apenas é preciso definir um formato standard de como os dados devem ser armazenados nas \textit{tags}, que podem ser NFC ou RFID. Este formato deve ser respeitado por todos os embaladores, assim, os produtos em vez de terem um código de barras, têm uma \textit{tag} RFID ou uma \textit{tag} NFC, com a informação necessária. Está fora do âmbito do nosso trabalho implementar o suporte hardware para a leitura das \textit{tags} e do sentido do movimento. Assume-sem que essa informação é disponibilizada num formato conhecido. 

A base de dados será desenhada como \textit{multi-tenant}, que pode aumentar os riscos de segurança da informação numa aplicação informática ligada em rede. Os riscos serão mitigados usando as boas práticas para proteção de dados, bem como de autorização nos acessos.

O projeto é composto por 5 blocos principais, inter relacionados entre si. A figura \ref{esquema_geral} representa esses blocos. Um dos blocos a desenvolver será a base de dados, para tal iremos usar o SGBD \textit{PostgreSQL}. Um outro bloco, denominado DAL, será necessário para o acesso à base de dados a fim de realizar leituras e escritas, fazendo uso da linguagem de programação \textit{Java}, com JDBC API. O bloco BLL servirá para gerir os dados obtidos através da base de dados ou da Web API. Para receber e enviar a informação, quer armazenada quer recolhida, iremos disponibilizar uma interface, acedida através da Web API. Vamos usar a \textit{framework} da \textit{Spring}, chamada de \textit{Spring Boot} para a implementação deste bloco. O último bloco será a interação com o utilizador, para tal iremos desenhar duas aplicações, uma móvel e uma Web. A aplicação móvel será apenas desenvolvida para a plataforma \textit{Android}, e vamos utilizar a linguagem \textit{Kotlin}. Para a aplicação Web, irá ser desenvolvida utilizando a linguagem \textit{JavaScript}, com o auxilio da \textit{framework Express}.

\subsection*{Riscos}
No caso de existir uma desistência de um elemento do grupo, o trabalho fica comprometido pois não será completado na sua totalidade. Se um elemento eventualmente desistir, cuja a responsabilidade seja a de realizar uma camada imprescindível para a realização do resto do trabalho, um outro elemento terá de prescindir das suas tarefas para realizar o trabalho do elemento que desistiu, caso contrário, o projeto não se desenvolveria. O risco seria as camadas superiores não ficarem realizadas na sua totalidade ou da forma mais correta.