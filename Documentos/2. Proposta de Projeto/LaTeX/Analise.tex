%
% Análise
%
\chapter*{Análise} \label{Analise}

Para a realização deste projeto encontrámos algumas dificuldades que se prendem com o facto de os produtos não terem rótulos digitais. Isto é um problema para a concretização do projeto, na medida em que se torna menos eficiente a obtenção dos dados presentes nos produtos. Contudo, assumindo que este dilema é resolvido fora do âmbito do projeto, apenas é preciso definir um formato standard de como os dados devem ser armazenados nas \textit{tags}, que podem ser NFC ou RFID. Formato este que deve ser respeitado por todos os produtores, assim, os produtos em vez de terem um código de barras, têm uma \textit{tag} RFID ou uma \textit{tag} NFC, com a informação necessária. E aplicando a responsabilidade da leitura das \textit{tags} nos dispositivos de hardware, e não fazendo este parte do campo de ação do nosso projeto, apenas necessitamos de gerir a informação transmitida pelos mesmos.

Como já deveríamos saber a segurança da informação nos dias que correm é um assunto com que todos nos devemos preocupar. E os dispositivos \textit{IoT} não fogem à regra. Desta forma temos de nos preocupar com a segurança dos dados pois se existir uma vulnerabilidade estamos a comprometer a casa e a segurança dos nossos utilizadores. Por estas razões óbvias temos de adotar funcionalidades de segurança ao nosso projeto.

O projeto será dividido em 5 camadas principais, inter relacionadas entre si. A camada mais acima é a das aplicações, uma mobile, para a plataforma \textit{Android} e usando a linguagem \textit{Kotlin}, e uma web, utilizando a linguagem \textit{JavaScript}, com a \textit{framework Express}. Abaixo encontra-se a Web API, para a construção desta faremos uso de uma \textit{framework} da \textit{Spring}, chamada de \textit{Spring Boot}. Esta ferramenta nunca foi utilizada, porém está a ser lecionada numa cadeira assistida por todos. A terceira camada, Camada da Lógica de Negócio (BLL), cuja separação da camada que a antecede é uma linha muito ténue. A Camada de Acesso a Dados, a segunda camada, será realizada com a linguagem de programação JAVA, com JDBC API, e é responsável por aceder à base dados com uma maior abstração, podendo-se escrever instruções SQL e obter os dados através de uma interface. Por último temos a Base de Dados, recorrendo ao SGBD \textit{PostgreSQL}, também ainda não usado.

No caso de existir uma desistência de um elemento do grupo, o trabalho fica comprometido pois não será completado na sua totalidade. Se um elemento eventualmente desistir, cuja a responsabilidade seja a de realizar uma camada imprescindível para a realização do resto do trabalho, um outro elemento terá de prescindir das suas tarefas para realizar o trabalho do elemento que desistiu, caso contrário, o projeto não se desenvolveria. O risco seria as camadas superiores não ficarem realizadas na sua totalidade ou da forma mais correta.