%
% Capítulo 1
%
\chapter*{Introdução} \label{cap1}

É indiscutível a quantidade de transtornos causados pela altura de reabastecer frigoríficos, armários e despensas. Por um lado, com elaboração da lista em si, que envolve uma longa e exaustiva verificação do que se tem e do que escasseia, como também aquando no supermercado e estabelecimentos nos deparamos com o esquecimento da lista de compras. Por outro lado, qual não seria o ganho se tudo fosse gerido de forma automática e simples, evitar-se-iam produtos esquecidos e expirados no fundo dos armários, ou a duplicação de produtos quando na verdade o que carecia era um outro tão indispensável. Uma aplicação capaz de tornar eficiente a gestão dos produtos do quotidiano. A simplificação deste problema, corresponderia a um ganho de compras e tempo, constrangimentos evitados e prateleiras abastecidas com o que é realmente necessário, sem gerar desperdícios.

Com a tendência crescente do mundo na automação e inteligência, consegue-se simplificar a gestão dos produtos armazenados em casa, tudo à distância de um clique no {\itshape smartphone} ou {\itshape tablet}, não menosprezando a web, com o uso de uma aplicação. Este projeto visa a interação entre uma aplicação móvel e outra web com dispositivos inteligentes, que enviam dados em tempo real. As informações dos produtos a armazenar em casa, são lidas por um leitor de {\itshape tags} NFC, {\itshape tags} estas programadas e aplicadas em sistemas de arrumação (caixas, sacos, etc.) para produtos avulsos ou assumindo a uniformidade internacional de rótulos, armazenados em {\itshape tags} NFC, num futuro próximo.

De forma geral, as funcionalidades básicas existentes serão rastrear as quantidades e validades dos produtos existentes, especificar limites mínimos de artigos indispensáveis, alertas de validades perto do fim da data, gerar a derradeira lista mensal ou semanal consoante a vontade do utilizador e a possibilidade dos utilizadores da mesma casa partilharem listas entre si.