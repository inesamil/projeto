% Classe do documento e parâmetros gerais.
\documentclass[tikz,a4paper]{standalone}

% Packages a utilizar e respetivos parâmetros.
\usepackage{pgfgantt}
\usepackage[portuguese]{babel}
\usepackage[utf8]{inputenc}

\title{Planeamento}

\begin{document}

%
% A fairly complicated example from section 2.9 of the package
% documentation. This reproduces an example from Wikipedia:
% http://en.wikipedia.org/wiki/Gantt_chart
%
\definecolor{barblue}{RGB}{153,204,254}
\definecolor{groupblue}{RGB}{51,102,254}
\definecolor{linkred}{RGB}{165,0,33}

\renewcommand\sfdefault{phv}
\renewcommand\mddefault{mc}
\renewcommand\bfdefault{bc}

\setganttlinklabel{s-s}{START-TO-START}
\setganttlinklabel{f-s}{FINISH-TO-START}
\setganttlinklabel{f-f}{FINISH-TO-FINISH}
\sffamily

\begin{ganttchart}[
    canvas/.append style={fill=none, draw=black!5, line width=.75pt},
    hgrid style/.style={draw=black!5, line width=.75pt},
    vgrid={*1{draw=black!5, line width=.75pt}},
    vgrid={*3{draw=none}, dotted},
    today=2,
    today rule/.style={
      draw=black!64,
      dash pattern=on 3.5pt off 4.5pt,
      line width=1.5pt
    },
    today label font=\small\bfseries,
    today label={\small\bfseries HOJE},
    title/.style={draw=none, fill=none},
    title label font=\bfseries\footnotesize,
    title label node/.append style={below=7pt},
    include title in canvas=false,
    bar label font=\mdseries\small\color{black!70},
    bar label node/.append style={left=2cm},
    bar/.append style={draw=none, fill=black!63},
    bar incomplete/.append style={fill=barblue},
    bar progress label font=\mdseries\footnotesize\color{black!70},
    group incomplete/.append style={fill=groupblue},
    x unit=0.55cm,
            y unit title=0.6cm,
            y unit chart=0.75cm,
    group left shift=0,
    group right shift=0,
    group height=.5,
    group peaks tip position=0,
    progress label text={#1\% completo},
    group label node/.append style={left=.6cm},
    group progress label font=\bfseries\small,
    link/.style={-latex, line width=1.5pt, linkred},
    link label font=\scriptsize\bfseries,
    link label node/.append style={below left=-2pt and 0pt}
  ]{1}{20}
 
 % Meses
  \gantttitle{Fev.}{1} \gantttitle{Mar.}{4}   \gantttitle{Abr.}{5} \gantttitle{Mai.}{4}   \gantttitle{Jun.}{4}   \gantttitle{Jul.}{2}\\
 % Semanas
  \gantttitlelist{1,...,20}{1} \\

 % Estudo do Problema 
  \ganttgroup[progress=90]{Estudo do problema}{1}{2} \\
  \ganttbar[progress=100]{Comparação das Tecnologias (\textit{NFC} vs \textit{RFID})}{1}{1} \\
  \ganttbar[progress=80]{Definição do Formato de Dados das \textit{Tags}}{1}{1} \\

 % Modelo de Dados
  \ganttgroup[progress=0]{Modelo de Dados}{2}{3}\\ 
  \ganttbar[progress=0]{Descrição do Problema}{2}{2} \\ 
  \ganttbar[progress=0]{Modelo EA}{2}{2} \\
  \ganttbar[progress=0]{Modelo Relacional}{2}{2} \\
  \ganttbar[progress=0]{Domínio dos Atributos}{2}{2}  \ganttnewline[grid]

 % MILESTONE: Proposta de Projeto
  \ganttmilestone[ name=M1]{\textbf{Entrega da Proposta de Projeto}}{4}  \ganttnewline[grid]

 % Base de Dados
  \ganttgroup[progress=0]{Base de Dados}{4}{5}\\  
  \ganttbar[progress=0]{Desenvolvimento}{4}{5}\\
   \ganttbar[progress=0]{Documentação}{5}{5} \\
   \ganttbar[progress=0]{Testes}{4}{5}  \\

 % Camada de Acesso a Dados
  \ganttgroup[progress=0]{Camada de Acesso a Dados}{6}{7}\\  
  \ganttbar[progress=0]{Desenvolvimento}{6}{7}\\
  \ganttbar[progress=0]{Documentação}{7}{7} \\
  \ganttbar[progress=0]{Testes}{6}{7} \\

 % Camada da Lógica de Negócio
   \ganttgroup[progress=0]{Camada da Lógica de Negócio}{8}{10}\\  
   \ganttbar[progress=0]{Desenvolvimento}{8}{10}\\
   \ganttbar[progress=0]{Documentação}{10}{10} \\
   \ganttbar[progress=0]{Testes}{8}{10}  \ganttnewline[grid]

 % MILESTONE: Relatório de Progresso
  \ganttmilestone[ name=M2]{\textbf{Entrega do Relatório de Progresso}}{10} \ganttnewline[grid]

  % WEB API
   \ganttgroup[progress=0]{WEB API}{11}{13}\\  
   \ganttbar[progress=0]{Desenvolvimento}{11}{13}\\
   \ganttbar[progress=0]{Documentação}{13}{13} \\
   \ganttbar[progress=0]{Testes}{11}{13}  \\

 % Aplicações (Android e Website)
   \ganttgroup[progress=0]{Aplicações (Android e Website)}{3}{15} \\
   \ganttbar[progress=0]{Desenho das User Interfaces}{3}{4}\\
   \ganttbar[progress=0]{Desenvolvimento}{14}{15}\\
   \ganttbar[progress=0]{Documentação}{15}{15} \\
   \ganttbar[progress=0]{Testes}{14}{15}  \ganttnewline[grid]

 % MILESTONE: Cartaz e Versão Beta
  \ganttmilestone[ name=M3]{\textbf{Entrega do Cartaz e Versão Beta}}{14} \ganttnewline[grid]

 % Escrita
   \ganttgroup[progress=0]{Escrita}{3}{20} \\
   \ganttbar[progress=60,name=S1A]{Proposta de Projeto}{1}{4} \\
   \ganttbar[progress=0,name=S1B]{Relatório de progresso}{4}{10} \\
   \ganttbar[progress=0,name=S5C]{Relatório Beta}{11}{14} \\
   \ganttbar[progress=0,name=S5D]{Relatório Final}{14}{20}\\
   \ganttbar[progress=0,name=S5E]{Desenvolvimento do cartaz}{12}{14} \ganttnewline[grid]
  
 % MILESTONE: Relatório de Progresso
  \ganttmilestone[ name=M4]{\textbf{Entrega do Projecto}}{20}  \ganttnewline[grid]

 % Ligações
  %\ganttlink{S1A}{M1}
  %\ganttlink{S1B}{M2}
  %\ganttlink{S1C}{M3}
 
\end{ganttchart}
\end{document}