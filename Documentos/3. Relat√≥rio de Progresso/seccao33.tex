%
% Acesso a Dados
%
\section{Acesso a Dados}\label{sec33}

Uma vez armazenados os dados de forma persistente é indispensável realizar escritas e leituras sobre os mesmos. Para tal, desenvolveu-se a chamada \acrfull{dal}. 

Para implementar esta camada, ponderaram-se duas opções, \gls{jpa} e \gls{jdbctemplate}. Apesar de \acrshort{jdbctemplate} permitir um maior controlo do lado do programador, não se fizeram notar discrepâncias significativas, pelo que se optou então por \acrshort{jpa}, por questões de familiaridade.

Como o modelo de dados é relativamente extenso (mais de 20 tabelas) o uso de \acrshort{jpa} torna-se benéfico. Tal, permite reduzir a extensa repetição de código envolvido para suportar as operações básicas de \acrfull{crud} em todas as entidades. 

Aqui, o requisito é o acesso aos dados na \acrshort{bd} e o suporte para as operações \acrshort{crud} em quase todas as tabelas. Desta forma criou-se uma interface \textit{Repository} com métodos que garantem não só essas operações, como outras para facilitar a obtenção de dados de determinada maneira. Existe ainda a possibilidade de criar \textit{queries}, definindo métodos nas interfaces \acrshort{jpa}. O uso de \acrshort{jpa} obriga a representar o esquema/modelo da \acrshort{bd} em classes \textit{Java}, \acrfull{pojo}.

 \subsection{Implementação}\label{subsec331}
 
 No acesso a dados, são utilizados dois padrões de desenho: Padrão \textit{Repository} e Padrão \textit{Unit Of Work}. Esta componente é, salvo exceções, gerada através da \acrshort{jpa}.
 
 Cada entidade presente na \acrshort{bd} é mapeada numa classe em Java, que representa o modelo da mesma. Esta classe tem várias anotações da JPA para referir a \acrlong{cp}, \acrlong{ce}, relações entre entidades, etc. Em conjunto estas classes Java formam o modelo utilizado entre as camadas internas do lado do servidor. Mais à frente serão apresentados outro tipos de objeto usados para representar as entidades recebidas e enviadas para o exterior.
 
 
