%
% Capítulo 2
%
\chapter{Formulação do Problema } \label{cap2}

Neste capítulo o problema é descrito de forma detalhada, bem como os requisitos funcionais e não funcionais. A secção \ref{sec23} apresenta as dificuldades que surgiram no decorrer do projeto. São expostas na secção \ref{sec24} noções da gestão de stocks 


%
% Secção 2.1
%
\section{Descrição do Problema} \label{sec21}
No âmbito do projeto existe a necessidade de criar um sistema de informação que permita gerir os produtos em stock de uma dada casa.

Uma casa está associada a um ou mais utilizadores, podendo um utilizador ter várias casas. Cada casa é caracterizada por um identificador único, um nome, atribuído por um utilizador no momento de registo da casa, quantos bebés, crianças, adultos e seniores vivem nessa casa. Cada utilizador é identificado univocamente por um email ou por um nome de utilizador, pelo nome da própria pessoa, idade e uma password. Para cada casa podem existir um ou mais administradores. Um utilizador pode criar as suas listas e as suas receitas. As listas que este cria pode decidir se quer partilhar com os restantes utilizadores da casa a que pertence, nunca podendo partilhar com utilizadores fora da sua casa. Existem listas que são comuns a todos os utilizadores registados, contudo são particulares a cada casa. As listas partilhadas pelos vários utilizadores dependem dos produtos que cada utilizador tem em casa. As receitas que um utilizador cria podem ser partilhadas com todos os utilizadores registados ou só com determinados utilizadores. Existem ainda um conjunto de receitas que são partilhadas por todos os utilizadores registados.

Cada receita é identificada por um identificador único, um nome, uma preparação, uma dificuldade, um tempo, quantas doses, qual o tipo de cozinha e qual o tipo de prato. Cada receita pode ter vários ingredientes. Para cada receita deve ser possível saber a quantidade dos vários ingredientes que a compõem.
Cada lista é identificada por um identificador único e um nome. Uma lista pode ter vários produtos. Para os produtos presentes numa lista pode ser possível saber a sua marca e a quantidade de um produto na lista. Um produto é identificado pelo seu identificador único, contém um nome, se é comestível ou não, e a validade perecível. Um produto pertence a uma categoria, podendo uma categoria ter vários produtos. Uma categoria é identificada por um identificador único ou um nome. Um produto pode ter vários itens em casa. Uma casa pode ter vários itens presentes na mesma. 

Um item presente numa casa é identificado por um identificador único ou por uma marca, uma variedade e um segmento, é também caracterizado por uma descrição, o local de conservação, a quantidade e as datas de validade. Para cada item deve ser possível saber os seus movimentos, isto é, se entrou ou saiu de um local de armazenamento. Para cada movimento deve ser possível saber o tipo de movimento (entrada ou saída), a data em que ocorreu o movimento e a quantidade de produtos que ocorrem num movimento.
Para cada casa existem vários locais de armazenamento dos itens, por exemplo armários, frigoríficos, etc. Cada local de armazenamento é caraterizado por um identificador único, a temperatura e um nome. Um local de armazenamento pode ter vários itens presentes numa casa e vários movimentos. Para cada local de armazenamento deve ser possível saber a quantidade de cada item.

Para cada casa deve ser possível saber que alergias os seus membros têm e para cada alergia saber o número de membros que têm essa alergia (os membros não precisam necessariamente de estar registados). Deve também ser possível saber os alergénios de um item presente na casa.



%
% Secção 2.2
%
\section{Requisitos Funcionais e Não Funcionais} \label{sec22}

\subsection{Requisitos Funcionais}
\begin{itemize}
	\item Informar o utilizador dos produtos existentes, a sua validade e a sua quantidade;
	\item Alertas sobre os produtos que estão perto da data de validade;
	\item Geração da lista de compras com os produtos em falta;
	\item Possibilidade de especificar os produtos a ter sempre em stock bem como as suas quantidades mínimas;	
	\item Lista de Compras Offline (permite rasurar para uso no supermercado);
	\item Listas partilhadas entre utilizadores da mesma casa;
	\item Criação de Listas (As Minhas Listas);
	\item Especificação das alergias dos membros da casa.
\end{itemize}


\subsection{Requisitos Não Funcionais}
\begin{itemize}
	\item Lista de produtos quase a expirar;
	\item Lista de produtos idesejados (Lista Negra);
	\item Lista de contenção em situações de emergência (Lista SOS);
	\item Sugestão de receitas que utilizem os produtos mais perto do fim da validade;
	\item Inserção de receitas (As Minhas Receitas);
	\item Inserir refeições extraordinárias de eventos a realizar num futuro próximo, para acrescentar alimentos não básicos à lista de compras;
	\item Especificação dietas alimentares (Vegetarianos, \textit{Vegans}, etc.) a cada utilizador da casa.
\end{itemize}


%
% Secção 2.3
%
\section{Dificuldades Encontradas} \label{sec23}

Para a realização deste projeto encontrámos dificuldades nos aspetos a seguir referidos.

\subsection{Rótulos em Formato Não Digital}

Nos dias de hoje, os produtos não possuem rótulos digitais. Isto é um problema para a concretização do projeto, na medida em que se torna menos eficiente a recolha dos dados presentes nos produtos. Contudo, assumindo que este dilema é resolvido fora do âmbito do projeto, apenas é preciso definir um formato standard de como os dados devem ser armazenados nas \textit{tags}, que podem ser \acrshort{nfc} ou \acrshort{rfid}. Num cenário ideal, este formato deve ser respeitado por todos os embaladores. Assim, os produtos têm um rótulo, código de barras e uma \textit{tag} \acrshort{nfc} ou \acrshort{rfid}, com a informação necessária. Está fora do âmbito do trabalho implementar o suporte hardware para a leitura das \textit{tags} e qual o sentido do movimento (entrada ou saída). Assume-se que essas informações são disponibilizadas num formato conhecido. 


\subsection{Ausência de Identificador Único nos Itens}

Os itens não dispõem de um identificador unívoco, alguns deles contêm um lote e um número de série. A ausência deste identificador impede a distinção entre itens iguais, o que impossibilita saber se entrou um novo item no local de armazenamento ou se saiu um dos itens presentes. Tal facto torna a gestão dos stocks dependente do dispositivo de hardware para distinguir o tipo de movimento.


%
% Secção 2.4
%
\section{Conceitos Básicos de Gestão de Stocks} \label{sec24}
	
\vspace{0.2cm}
\textbf{Inventário} - Um catálogo detalhado ou uma lista de bens ou propriedades tangíveis, ou os atributos ou qualidades intangíveis.

%Read more: http://www.businessdictionary.com/definition/inventory.html

\vspace{0.2cm}
\textbf{\acrfull{sku} (Unidade de Manutenção de Stock, em Português)} - Um código de identificação de um produto e serviço para uma loja ou produto, muitas vezes retratado como um código de barras legível por máquinas que ajuda a rastrear o item para inventários. Ver exemplo 1.

\exampleblock{
Por exemplo, um armário pode ter  pacotes de leite magro da marca X, 2 pacotes de leite magro da marca Y e 1 pacote de leite meio gordo da marca X. Logo, o armário contém 3 \acrshort{sku}, uma vez que um \acrshort{sku} se distingue pelo tamanho, cor, sabor, marca, etc.}

%https://www.investopedia.com/terms/s/stock-keeping-unit-sku.asp

\vspace{0.2cm}
\textbf{Stock Item (Item de Stock, em Português)} - Refere-se aos itens que se mantêm em stock físico na loja. O item de stock tem uma quantidade associada. Cada vez que uma venda é feita para aquele item, a sua quantidade será deduzida. 
Artigo aprovado para aquisição, armazenamento e emissão, e geralmente mantido à mão.

%Read more: http://www.businessdictionary.com/definition/standard-stock-item.html

%http://support.phostersoft.com/support/solutions/articles/17907-stock-item-and-non-stock-item

\vspace{0.2cm}
\textbf{Product Category (Categoria de Produtos, em Português)} - Taxonomias de classificação que subdividem um Setor ("yet another market construct") nos diferentes tipos de produtos para os quais existe demanda. Quanto mais especializada for uma categoria, mais especializado é o produto. 

{\footnotesize Nota: Neste projeto apenas se consideram as categorias de maior dimensão, são elas, por exemplo, Laticínios, Bebidas, Frescos, Congelados, entre outras.}

%https://sphereoi.com/studios/category-segment-and-brand-whats-the-difference/

\vspace{0.2cm}
\textbf{Brand (Marca, em Português)} - Um símbolo de identificação, marca, logótipo, nome, palavra e/ou frase que as empresas usam para distinguir seus produtos dos outros.

%Read more: Brand https://www.investopedia.com/terms/b/brand.asp#ixzz5Al2FpNL2
 
\vspace{0.2cm}
\textbf{Segmentation (Segmento, em Português)} - Quando os estrategistas de marca falam sobre segmento,referem-se à segmentação do consumidor/audiência. A maneira antiga de abordar isso era através da demografia (idade, sexo, etnia, faixa de renda, urbano-rural, etc.). Agora a segmentação é VALS (valores, atitudes e estilo de vida). 

{\footnotesize Nota: Neste projeto o segmento é a quantidade presente numa embalagem, i.e., para um pacote de leite de 1L, o segmento é 1L.}

%https://sphereoi.com/studios/category-segment-and-brand-whats-the-difference/

\vspace{0.2cm}
\textbf{Variety (Variedade, em Português)} - A variedade é confusa porque pode ser difícil entender onde a especialização da segmentação termina e a especialização em prol da Variedade começa. A variação é sobre a personalização de um produto para se adequar ao caráter do consumidor individual. Ver exemplo 2.

%https://sphereoi.com/studios/category-segment-and-brand-whats-the-difference/
\vspace{0.5cm}
\exampleblock{
	Note-se um pacote de leite com as caraterísticas, quantidade líquida igual a 1L, da marca X e do tipo UHT magro. Então, identificar-se-ia da seguinte forma: 
	\begin{itemize}
		\item Categoria: Laticínios
		\item Produto: Leite
		\item Marca: X
		\item Segmento: 1L
		\item Variedade: UHT Magro
\end{itemize}}




