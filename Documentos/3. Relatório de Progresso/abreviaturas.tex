\label{abreviaturas}
 
 % Tecnologias
\newacronym{nfc}{NFC}{Near-Field Communication}
\newacronym{rfid}{RFID}{Radio-frequency Identification}
\newacronym{iot}{IoT}{Internet of Things}

% Ferramentas
\newacronym{sgdb}{SGDB}{Sistema de Gestão de Base de Dados}
\newacronym{jpa}{JPA}{Java Persistent API}
\newacronym{jdbc}{JDBC}{Java Database Connectivity}

% Camadas
\newacronym{db}{DB}{Database}
\newacronym{dal}{DAL}{Data Acess Layer}
\newacronym{bll}{BLL}{Business Logic Layer}

% Formato
\newacronym{json}{JSON}{JavaScript Object Notation}

% Gestão de Stocks
\newacronym{sku}{SKU}{Stock Keeping Unit}

% Modelo Relacional
\newacronym{cp}{CP}{Chave-Primária}
\newacronym{ce}{CE}{Chave-Estrangeira}
\newacronym{occ}{OCC}{Outra Chave-Candidata}


% \acrlong{ } 
% Displays the phrase which the acronyms stands for. Put the label of the acronym inside the braces. In the example, \acrlong{gcd} prints Greatest Common Divisor.

% \acrshort{ } 
%Prints the acronym whose label is passed as parameter. For instance, \acrshort{gcd} renders as GCD.

%\acrfull{ } 
%Prints both, the acronym and its definition. In the example the output of \acrfull{lcm} is Least Common Multiple (LCM).