% Página de resumo em Português
\cleardoublepage\newpage
\chapter*{Resumo} \label{resumo}
A gestão de stocks ajuda a administrar, de forma otimizada, os investimentos em stock de uma casa. Com um sistema de gestão apropriado à casa é possível obter diversos benefícios, tais como, acabar com o esquecimento de produtos no fundo dos locais de armazenamento ou o de produtos fora da validade.
O objetivo da gestão de stocks envolve a determinação de três principais decisões:
\begin{itemize}
	\item quando comprar os produtos,
	\item quantidade a comprar dos produtos e
	\item stock mínimo de segurança que se deve manter para cada produto.
\end{itemize}
Estas deliberações assumem uma dinâmica estruturada e repetitiva ao longo do tempo.
Desta forma, pretende-se desenvolver um sistema de gestão de stocks que inclui uma aplicação móvel e web, com suporte inteligente de um algoritmo de previsão de stocks. Tem-se por base a automatização da recolha de dados, com o uso de sensores. Este sistema simplifica não só, o controlo de stocks, como também, a análise dos padrões de consumo e reposição de uma casa. Assim consegue-se auxiliar os utilizadores a manter o stock adequado às suas necessidades, bem como alertá-los para a proximidade do fim da validade e/ou stock dos produtos.

\vspace{0.2cm}
{\bf Palavras-chave: Administração; Automatização; Controlo; Gestão; Previsão; Sensores; Sistema; Stock;} 