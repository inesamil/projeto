%
% Capítulo 1
%
\chapter{Terminologia} \label{a1}

%
% Secção A.1
%
\section{Conceitos Básicos de Gestão de Stocks} \label{seca11}
% Contador de Exemplos
\newcounter{ExampleCounter}

\vspace{0.2cm}
\textbf{Inventário} - Um catálogo detalhado ou uma lista de bens ou propriedades tangíveis, ou os atributos ou qualidades intangíveis. Ler mais em \cite{businessDictionary:invetoryDefinition2018}.

\vspace{0.2cm}
\textbf{\acrfull{sku} (Unidade de Manutenção de Stock, em Português)} - Um código de identificação de um produto e serviço para uma loja ou produto, muitas vezes retratado como um código de barras legível por máquinas que ajuda a rastrear o item para inventários. Ver exemplo \ref{seca11}.1. Ler mais em \cite{investopedia:skuDefinition2018}.

\stepcounter{ExampleCounter}
\noindent\fbox{
	\parbox{\textwidth}{
		\textbf{Exemplo \arabic{ExampleCounter}}\\
		Por exemplo, um armário pode ter  pacotes de leite magro da marca X, 2 pacotes de leite magro da marca Y e 1 pacote de leite meio gordo da marca X. Logo, o armário contém 3 \acrshort{sku}, uma vez que um \acrshort{sku} se distingue pelo tamanho, cor, sabor, marca, etc.
	}
}

\vspace{0.2cm}
\textbf{Stock Item (Item em Stock, em Português)} - Refere-se aos itens que se mantêm em stock físico na loja. O item de stock tem uma quantidade associada. Cada vez que uma venda é feita para aquele item, a sua quantidade será deduzida. 
Artigo aprovado para aquisição, armazenamento e emissão, e geralmente mantido à mão. Ler mais em \cite{businessDictionary:stockItemDefinition2018} e \cite{phostersoft:stockItemDefinition2018}.


\vspace{0.2cm}
\textbf{Product Category (Categoria de Produtos, em Português)} - Taxonomias de classificação que subdividem um Setor ("yet another market construct") nos diferentes tipos de produtos para os quais existe demanda. Quanto mais especializada for uma categoria, mais especializado é o produto. Ler mais em \cite{sphereoi:itemIdentification2018}.

{\footnotesize Nota: Neste projeto apenas se consideram as categorias de maior dimensão, são elas, por exemplo, Laticínios, Bebidas, Frescos, Congelados, entre outras.}

\vspace{0.2cm}
\textbf{Brand (Marca, em Português)} - Um símbolo de identificação, marca, logótipo, nome, palavra e/ou frase que as empresas usam para distinguir os seus produtos dos outros. Ler mais em \cite{investopedia:brandDefinition2018}

\vspace{0.2cm}
\textbf{Segmentation (Segmento, em Português)} - Quando os estrategistas de marca falam sobre segmento,referem-se à segmentação do consumidor/audiência. A maneira antiga de abordar isso era através da demografia (idade, sexo, etnia, faixa de renda, urbano-rural, etc.). Agora a segmentação é VALS (valores, atitudes e estilo de vida). Ler mais em \cite{sphereoi:itemIdentification2018}.

{\footnotesize Nota: Neste projeto o segmento é a quantidade presente numa embalagem, i.e., para um pacote de leite de 1L, o segmento é 1L.}


\vspace{0.2cm}
\textbf{Variety (Variedade, em Português)} - A variedade é confusa porque pode ser difícil de entender onde a especialização da segmentação termina e a especialização em prol da Variedade começa. A variação é sobre a personalização de um produto para se adequar ao caráter do consumidor individual. Ver exemplo \ref{seca11}.2. Ler mais em \cite{sphereoi:itemIdentification2018}.

\vspace{0.5cm}

\stepcounter{ExampleCounter}
\noindent\fbox{
	\parbox{\textwidth}{
		\textbf{Exemplo \arabic{ExampleCounter}}\\	
		Note-se um pacote de leite com as caraterísticas, quantidade líquida igual a 1L, da marca X e do tipo UHT magro. Então, identificar-se-ia da seguinte forma: 
		\begin{itemize}
			\item Categoria: Laticínios
			\item Produto: Leite
			\item Marca: X
			\item Segmento: 1L
			\item Variedade: UHT Magro
		\end{itemize}
	}
}

%
% Secção A.2
%
%\section{Conceitos de ???} \label{seca12}	
	
%\glsaddall
%\printglossary[type=\acronymtype,title=Acrónimos]
