%
% Capítulo 1
%
\chapter{Introdução} \label{cap1}

A gestão de stocks é uma tarefa estruturada e repetitiva, para a qual já existem soluções capazes de fornecer listas de compras. \textit{OutOfMilk} e \textit{Bring} são exemplos dessas soluções no formato de aplicações \textit{mobile}. Contudo carecem de controlo de stocks e conhecimento dos hábitos dos seus utilizadores.  Como tal, por meio de uma aplicação \textit{mobile} e \textit{web} com suporte inteligente de um algoritmo de previsão de stocks pretende-se solucionar este problema.

Tendo por base a automatização da recolha de dados, simplifica-se, não só, o controlo de stocks, como também, a análise dos padrões de consumo e reposição numa casa.
Desta forma, auxilia-se os utilizadores a manter o stock adequado às suas necessidades, bem como alertá-los para a proximidade do fim da validade e/ou stock dos produtos. 

Assim, este trabalho vai no sentido de responder a questões como: ``De que forma podemos evitar transtornos causados na altura de reabastecer a nossa despensa? Ou como proceder ao controlo de stocks de alimentos e outros produtos? E como impedir artigos fora de prazo?". Se se entender que uma casa funciona como uma empresa e existem quantidades mínimas recomendadas, é possível gerar uma nota de encomenda com os produtos em falta ou prestes a terminar para o utilizador poder consultar e exercer a compra.

%
% Secção 1.1
%
\section{Contexto} \label{sec11}

Uma boa gestão de stocks de mercadorias é de extrema importância porque tem reflexos imediatos nos resultados de uma empresa, o que permite manter os clientes satisfeitos não só a nível da quantidade como da qualidade. Para manter o stock ideal não basta bom senso e intuição, é necessário conhecer o fluxo de vendas, utilizar ferramentas adequadas de gestão de informação sobre movimentos e eventuais constrangimentos no fornecimento. Extrapolando para a empresa "casa", o processo é apenas um problema de escala. Organizar a despensa como se de uma empresa se tratasse possibilita uma melhor logística de custos e tempo. Ao elaborar uma lista de stock, onde se vai anotando os produtos que se tem, o que está a acabar e o que se tem de comprar, passa por uma solução indispensável. Por vezes torna-se numa tarefa que ``não é para todos".

Perante este problema, pretende-se desenvolver um projeto, utilizando uma solução digital, aplicação \textit{mobile} e de \textit{web}, que tem como objetivo ajudar os portugueses nesta repetitiva tarefa que é adotar e manter, ao longo do tempo, a sua despensa sem faltas.
Através da mesma, o individuo terá sempre presente informação útil e prática, com possibilidade de utilizar um formato de lembretes e de registar as tendências para uma futura investigação no que diz respeito aos hábitos de consumo.

Destaca-se ainda o facto de, no contexto atual, existir um aumento na facilidade de acesso às novas tecnologias, nomeadamente à \textit{internet}. Em plena era da informação a proliferação dos meios de comunicação e da própria \textit{internet} permitiu que os utilizadores se liguem à rede 24 horas, por dia, através de telemóveis, portáteis, \textit{tablets} e outros. A cada dia que passa assiste-se a uma mudança massiva do comportamento do consumidor nesta área, graças à utilização dos dispositivos móveis dos ``8 aos 80" anos. Conforme os dados divulgados em Dezembro de 2016 pelo Gabinete de Estatísticas da União Europeia (Eurostat) \cite{eurostat:internetAccess2016}.

%
% Secção 1.2
%
\section{Metas e Objetivos} \label{sec12}
Face ao exposto, a existência de uma aplicação móvel na agenda de tarefas de uma organização doméstica poderá ser uma mais valia. Para o concretizar de uma aplicação móvel foi necessário cumprir com objetivos mais específicos que respondessem à seguinte questão: 

``Quais as características e funcionalidades que deverá ter a aplicação que sejam úteis para os utilizadores e se diferencia das restantes?"

Deste modo, definiram-se os seguintes objetivos:
\begin{itemize} \itemsep 0pt
	\item Rever e sumarizar os conteúdos das aplicações móveis mais populares e com classificação mais elevada.
	\item Criar o primeiro protótipo de uma aplicação simples e educativa com base em estratégias de usabilidade e manutenção.
	\item Avaliar a rentabilidade e sustentabilidade do negócio resultante do lançamento no mercado de uma aplicação com as descrições acima apresentadas.
\end{itemize}

%
% Secção 1.3
%
\section{Abordagem do Projeto} \label{sec13}

Este trabalho divide-se em duas partes principais. A primeira com o enquadramento teórico, em que se fez uma revisão da literatura focando os principais temas associados ao projeto, nomeadamente, gestão de stocks, a utilização das novas tecnologias. Reviu-se também estratégias de usabilidade e promoção de literatura na construção das aplicações móveis bem como a regulamentação existente e possibilidade de certificação.
Ainda nesta parte, efetuou-se investigação exploratória de suporte à elaboração do projeto, assim como análise e discussão dos resultados obtidos.

Na segunda parte encontra-se o trabalho desenvolvido para o projeto, ou seja os requisitos do projeto, a solução implementada e acesso a dados. Concluindo com o plano de negócio, e.g. apresentação da empresa e do mercado, objetivos e estratégia da empresa.


%
% Secção 1.4
%
\section{Estrutura do Relatório} \label{sec14}
O relatório está estruturado em 9 capítulos.

O capítulo 2 formula o problema, detalhando os requisitos do projeto. 

No capítulo 3 o problema é solucionado, sendo apresentada a solução implementada. É ainda apresentada uma análise da solução.

Os casos de uso são expostos, com detalhe, no capítulo 4.

No capítulo 5 são abordadas as aplicações de interação direta com o utilizador.

O capítulo 6 destina-se ao esclarecimento do desenvolvimento e implementação do algoritmo de previsão.

É no capítulo 7 que é feita referência à \gls{api-web} bem como todas as suas particularidades.

O capítulo 8 elucida a modelagem dada aos dados. Por conseguinte explicita de que forma esses dados foram armazenados, sendo ainda justificadas as decisões tomadas nesta camada. A lógica de negócio está presente, também, neste capítulo. 

É no capítulo 9 que se retiram conclusões face ao trabalho desenvolvido em relação ao trabalho inicialmente previsto. Este planeamento inicial é revelado na secção \ref{sec41}. Para finalizar, propõe-se o trabalho a realizar futuramente, na secção \ref{sec43}.

No Anexo A define-se terminologia, quer a básica à gestão de stocks, para melhor compreensão de alguns dos termos utilizados no decorrer do projeto, quer de conceitos de programação.

