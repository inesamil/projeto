%
% Capítulo 1
%
\chapter{Introdução} \label{cap1}

A gestão de stocks é uma tarefa estruturada e repetitiva, para a qual já existem soluções capazes de fornecer listas de compras. \textit{OutOfMilk}\footnote[1]{https://www.outofmilk.com/} e \textit{Bring}\footnote[2]{https://www.getbring.com/\#!/app} são exemplos dessas soluções no formato de aplicações \textit{mobile}. Contudo carecem de controlo de stocks e conhecimento dos hábitos dos seus utilizadores.  Como tal, por meio de uma aplicação \textit{mobile} e \textit{web} com suporte inteligente de um algoritmo de previsão de stocks pretende-se solucionar este problema.

Tendo por base a automatização da recolha de dados recorrendo a sensores, simplifica-se, não só, o controlo de stocks, como também, a análise dos padrões de consumo e reposição numa casa.
Desta forma, auxilia-se os utilizadores a manter o stock adequado às suas necessidades, bem como alertá-los para a proximidade do fim da validade e/ou stock dos produtos. 

Assim, este trabalho vai no sentido de responder a questões como: ``De que forma podemos evitar transtornos causados na altura de reabastecer a nossa despensa? Ou como proceder ao controlo de stocks de alimentos e outros produtos? E como impedir artigos fora de prazo?". Se se entender que uma casa funciona como uma empresa e existem quantidades mínimas recomendadas, é possível gerar uma nota de encomenda com os produtos em falta ou prestes a terminar para o utilizador poder consultar e exercer a compra.

%
% Secção 1.1
%
\section{Contexto} \label{sec11}

Uma boa gestão de stocks de mercadorias é de extrema importância porque tem reflexos imediatos nos resultados de uma empresa, o que permite manter os clientes satisfeitos não só a nível da quantidade como da qualidade. Para manter o stock ideal não basta bom senso e intuição, é necessário conhecer o fluxo de vendas, utilizar ferramentas adequadas de gestão de informação sobre movimentos e eventuais constrangimentos no fornecimento. Extrapolando para a empresa ``casa", o processo é apenas um problema de escala. Organizar a despensa como se de uma empresa se tratasse possibilita uma melhor logística de custos e tempo. Ao elaborar uma lista de stock, onde se vai anotando os produtos que se tem, o que está a acabar e o que se tem de comprar, passa por uma solução indispensável. Que por vezes se torna numa tarefa que ``não é para todos".

Perante este problema, pretende-se desenvolver um sistema, utilizando uma solução digital, aplicação \textit{mobile} e de \textit{web}, que tem como objetivo ajudar os portugueses nesta repetitiva tarefa que é adotar e manter, ao longo do tempo, a sua despensa sem faltas.
Através desta solução, o individuo terá sempre presente informação útil e prática, com possibilidade de utilizar um formato de lembretes e de registar as tendências para uma futura investigação no que diz respeito aos hábitos de consumo.

Destaca-se ainda o facto de, no contexto atual, existir um aumento na facilidade de acesso às novas tecnologias, nomeadamente à \textit{internet}. Em plena era da informação a proliferação dos meios de comunicação e da própria \textit{internet} permitiu que os utilizadores se liguem à rede 24 horas, por dia, através de telemóveis, portáteis, \textit{tablets} e outros. A cada dia que passa assiste-se a uma mudança do comportamento do consumidor nesta área, graças à utilização dos dispositivos móveis dos ``8 aos 80" anos. Conforme os dados divulgados em Dezembro de 2016 pelo Gabinete de Estatísticas da União Europeia (Eurostat) \cite{eurostat:internetAccess2016}.

%
% Secção 1.2
%
\section{Metas e Objetivos} \label{sec12}
Face ao exposto, a existência de um sistema de gestão de stocks na agenda de tarefas de uma organização doméstica poderá ser uma mais valia. Para concretizar esse sistema foi necessário cumprir com objetivos mais específicos que respondessem à seguinte questão: 

``Quais as características e funcionalidades que deverá ter o sistema que sejam úteis para os utilizadores e se diferencia das restantes?"

Deste modo, definiram-se os seguintes objetivos:
\begin{itemize} \itemsep 0pt
	\item Implementar uma interface com o utilizador para dispositivos móveis;
	\item Implementar uma interface com o utilizador para dispositivos \textit{desktop};
	\item Implementar a componente servidora de um sistema de gestão de stocks;
	\item Implementar um algoritmo de previsão de stocks.
\end{itemize}

%
% Secção 1.3
%
\section{Abordagem do Projeto} \label{sec13}

Este trabalho divide-se em duas partes principais. A primeira com o enquadramento teórico, em que se fez uma revisão da literatura focando os principais temas associados ao projeto, nomeadamente, gestão de stocks, a utilização das novas tecnologias. Reviu-se também estratégias de usabilidade e promoção de literatura na construção das aplicações móveis bem como a regulamentação existente e possibilidade de certificação.
Ainda nesta parte, efetuou-se investigação exploratória de suporte à elaboração do projeto, assim como análise e discussão dos resultados obtidos.

Na segunda parte encontra-se todo o trabalho desenvolvido para o projeto. Inicialmente definiram-se os requisitos fundamentais ao sistema de gestão de stocks a desenvolver. Aqui, foi também importante comparar as funcionalidades que se pretendiam implementar com as de outros sistemas já existentes, de forma a garantir elementos inovadores na solução. Posteriormente, definiu-se a arquitetura do sistema tal como todas as partes envolvidas. Foi ainda necessário estabelecer o modo como o utilizador iria interagir com o sistema, através do desenho das interfaces gráficas.

%
% Secção 1.4
%
\section{Estrutura do Relatório} \label{sec14}
O relatório está estruturado em 5 capítulos.

O capítulo 2 introduz o sistema de gestão de stocks desenvolvido, como também, formula o problema, detalhando os requisitos do projeto e casos de uso. 

No capítulo 3 o problema é solucionado, sendo apresentada a solução implementada. É ainda efetuada uma análise desta.

No capítulo 4 são abordados, em secções, as aplicações de interação direta com o utilizador, o desenvolvimento e implementação do algoritmo de previsão de stocks, a \gls{api-web} bem como todas as suas particularidades e a modelagem dada aos dados. Explica-se de que forma esses dados foram armazenados, sendo ainda justificadas as decisões tomadas. 

É no capítulo 5 que se retiram conclusões face ao trabalho desenvolvido em relação ao trabalho inicialmente previsto. Para finalizar, propõe-se o trabalho a realizar futuramente, na secção ???.

No Anexo A define-se terminologia, quer a básica à gestão de stocks, para melhor compreensão de alguns dos termos utilizados no decorrer do projeto, quer de conceitos de programação.

