% Página de resumo em Português
\cleardoublepage\newpage
\chapter*{Resumo} \label{resumo}
A gestão de stocks ajuda a controlar, de forma otimizada, os investimentos em stock de uma casa. Com um sistema de gestão apropriado é possível obter diversos benefícios, tais como, acabar com o esquecimento de produtos no fundo dos locais de armazenamento ou fora da validade.
A gestão de stocks envolve três decisões principais:
\begin{itemize}
	\item decidir quando comprar os produtos,
	\item determinar a quantidade a comprar de cada produto e,
	\item garantir um stock mínimo de segurança para cada produto.
\end{itemize}
Estas decisões assumem uma dinâmica que se repete ao longo do tempo. Pretende-se desenvolver um sistema de gestão automática de stocks, onde a recolha de informação é feita por sensores. O sistema é constituído, para além dos sensores, por uma aplicação móvel e web, e um servidor que implementa um algoritmo de previsão de stocks e que disponibiliza uma \gls{api-web}. Este sistema simplifica não só o controlo de stocks, como também a análise dos padrões de consumo e reposição de uma casa. Assim consegue-se auxiliar os utilizadores a manter o stock adequado às suas necessidades, bem como alerta-los para a proximidade do fim da validade e/ou stock dos produtos.

\vspace{0.2cm}
{\bf Palavras-chave: IoT; Gestão de Stock; Previsão de consumo; Sensorização em casa} 