%
% Capítulo 4
%
\chapter{Conclusões} \label{cap4}

Neste capítulo apresentam-se as conclusões relativas ao desempenho e trabalho realizado pelo grupo e, ainda, as diretrizes relativas a trabalho futuro. 

%
% Secção 4.1 Sumário
%
\section{Sumário}\label{sec41}

A possibilidade de visualização em tempo real do estado do stock de uma casa é um auxílio enorme.
No entanto, transmitir essa informação de forma simples, clara e percetível aos utilizadores é requerente de um prévio trabalho de pesquisa, para se poder conhecer o que os utilizadores procuram, de maneira a oferecer serviços adequados às suas necessidades, tornando-os acessíveis e dinamiza-los de forma simples e prática.
Com a finalidade de implementação do sistema Smart Stocks usou-se um modelo de dados para o armazenamento dos dados indispensáveis à gestão de stocks. Implementou-se uma \gls{api-web}, independente do modelo de dados e das aplicações cliente, responsável por disponibilizar os dados às aplicações clientes, móvel e web. A aplicação móvel é responsável, não só, por permitir a visualização da informação fornecida pela \gls{api-web}, tal como a aplicação web, como também, por interagir com o sistema e escrever nas \textit{tags}.

Realizaram-se testes à resistência das \textit{Tags} NFC e testes unitários às funções utilitárias desenvolvidas. De forma a realizar testes ao sistema Smart Stocks num ambiente real, desenvolveu-se uma aplicação para dispositivos móveis \textit{Android}, cuja principal função é simular um dispositivo de \textit{hardware}, como os que estão presentes em armários e frigoríficos que integram o sistema desenvolvido neste projeto.

%
% Secção 4.2 Trabalho Futuro
%
\section{Trabalho Futuro}\label{sec42}

As principais vertentes para um trabalho futuro centram-se em adquirir um certificado para API, de modo a que o sistema possa ir para produção. Igualmente importante, é a aquisição de funcionalidades de administração do sistema Smart Stocks para o papel \textit{ADMIN}, de forma a criar uma aplicação de \textit{back-office}, que se ocupasse da gestão das categorias e produtos, multi-línguas, adição de novas listas de sistema, assim como, novas funcionalidades e/ou regras de negócio. Melhorar a gestão dos utilizadores de uma casa, de maneira, por exemplo, para quando um utilizador sair de uma casa, o administrador receber uma notificação e ter de verificar e atualizar os dados básicos da casa, como o caso das alergias e do número de pessoas alérgicas da casa. Realizar um sistema de cores que nas aplicações cliente serviria como um aviso visual do estado dos itens em stock, tanto para a sua quantidade, como para a sua validade, aproveitando a análise dos resultado do algoritmo de previsão de stocks. Seria ainda interessante, permitir que o utilizador definisse os seus próprios limites mínimos dos produtos a ter em stock de maneira a tornar o sistema mais pessoal a cada pessoa e a cada casa. Outra melhoria a incluir, seria a criação de um sistema de notificações, de modo a deixar os utilizador sempre informados de qualquer alteração que aconteça, fossem elas por \textit{e-mail} ou recorrendo às notificações do \textit{Android}. E ainda, realizar testes unitários que testem a grande maioria das particularidades e casos. Por fim, uma melhoria à qualidade da segurança oferecida, de forma a garantir a confidencialidade máxima dos dados dos utilizadores e das suas casas.
