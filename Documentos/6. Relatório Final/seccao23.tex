%
% Secção 2.3
%
\section{Arquitetura da Solução}\label{sec23}

Nesta secção pretende-se abordar de forma geral a solução implementada para resolver o problema apresentado no capítulo \ref{cap1}.

%
% Subsecção 2.3.1 Abordagem
%
\subsection{Abordagem}\label{subsec231}

\begin{figure}[H]
	\centering
	\includegraphics[width=16cm, height=8cm, scale=1]{img/architecture.png}
	\caption{Arquitetura Geral do Projeto}
	\label{project-general-architecture}
\end{figure}

Após uma ida às compras, os itens adquiridos com rótulos não tradicionais, Figura \ref{project-general-architecture}(a), são armazenados nos seus respetivos locais, Figura \ref{project-general-architecture}(d). Como forma de automatizar a recolha de informação relativa quer aos artigos obtidos quer às suas caraterísticas, utilizam-se rótulos digitais e sensores.

Ao guardar os artigos nos locais de armazenamento, os seus rótulos devem ser lidos por dispositivos de hardware, conjunto sensor mais leitor de rótulos digitais, presentes no local, de forma a que a informação e identificação do item, bem como, o tipo de movimento (entrada ou saída) possam ser enviados para a componente servidora, Figura \ref{project-general-architecture}(e). Assim, estes dados são posteriormente tratados e armazenados de forma persistente na \acrfull{bd}, Figura \ref{project-general-architecture}(f). A componente servidora é responsável por retornar dados para as aplicações cliente, Figura \ref{project-general-architecture}(b, c). É ainda nesta que está presente o algoritmo de previsão de stocks utilizado para efetuar a previsão quanto à duração de cada um dos itens em stock, assim como, o controlo da gestão de stocks.

No contexto da gestão de stocks assume-se a existência de duas formas de apresentação para os itens em stock: avulsos e embalados. Os primeiros são conservados em sistemas de arrumação identificados com \textit{tags} programáveis por \textit{smartphones}, \ref{project-general-architecture}(c). Os detalhes dos itens são especificados pelo utilizador e carregados para a \textit{tag}. Já os segundos contêm os seus rótulos digitais com o detalhe guardado pelos embaladores.

%
% Tags
%
%
% Tags
%
\subsection{Rótulos dos Itens em Stock}
\subsubsection{Requisitos Legais dos Rótulos}
Segundo o Regulamento (UE) nº1169/2011 \cite{asae:labeling} os rótulos devem conter a seguinte lista de informação:
\begin{itemize}
    \item Denominação da venda;
    \item Listas de ingredientes;
    \item Quantidades de ingredientes ou das categorias de ingredientes;
    \item Quantidade líquida;
    \item Data de durabilidade mínima (DDM)/ Data limite de consumo (DLM);
    \item Condições especiais de conservação e utilização;
    \item Nome ou firma e endereço fabricante, do acondicionador ou do vendedor;
    \item País de origem ou de proveniência;
    \item Instruções de utilização;
    \item Referência ao teor alcoométrico volúmico adquirido.
\end{itemize}

\subsubsection{Rótulos Tradicionais \textit{vs} Rótulos Digitais}
Os códigos de barras são amplamente utilizados na identificação de produtos, quer seja dentro da própria organização, quer seja quando a empresa produtora pretende vender os seus produtos no mercado. Neste último caso, a codificação dos produtos, sendo uma obrigação do mercado, segue as normas da organização GS1\footnote[1]{https://www.gs1.org/}, organização responsável pelo sistema de Normas Globais de Identificação e Codificação de bens e serviços mais utilizado no mundo. A GS1 Portugal é a entidade competente geradora e reguladora da atribuição dos códigos de barras em Portugal. É garantido que não existem dois produtos com o mesmo código de barras em circulação quer a nível nacional, quer a nível global. 

São três os componentes relevantes identificados por um código de barras: o país de origem, a empresa fabricante e o produto produzido, ilustrado na Figura~\ref{bar-code-example} (um exemplo fictício).

\begin{figure}[H]
	\centering
	\includegraphics[scale=0.8]{img/codigo_barras_no_background.png}
	\caption{Exemplo proposto de código de barras}
	\label{bar-code-example}
\end{figure}

A gestão de stocks do sistema Smart Stocks necessita de saber o nome do produto, a marca, a variedade, o segmento, a data de validade, os alergénios, a quantidade e, opcionalmente, as condições de conservação. Logo, a informação presente nos códigos de barras é insuficiente para o correto funcionamento do sistema. Como tal, existiu a necessidade de encontrar uma nova abordagem que solucionasse o problema. Uma solução possível passa pela utilização de um leitor de imagens ou de objetos 3D, capaz de ler a informação presente no rótulo tradicional. No entanto, uma outra hipótese é a utilização de rótulos digitais, por exemplo, recorrendo a \textit{tags} \acrfull{nfc} \cite{nfcforum:nfc} ou \acrfull{rfid} \cite{rfidinc:rfid}. 
Como os produtos avulsos têm de ser rotulados, decidiu-se usar \textit{tags} programáveis por \textit{smartphones}. Este processo torna-se prático e acessível a muitos. Ora, uma vez que esta tecnologia é utilizada para os produtos avulsos e sendo uma das soluções possíveis para os produtos embalados, então, uniformiza-se a automatização utilizando a mesma tecnologia nas duas circunstâncias.

\subsubsection{Comparação entre \textit{Tags} NFC e RFID}
\begin{table}[H]
	\centering
	\caption{Comparação da tecnologia \acrshort{nfc} com a tecnologia \acrshort{rfid}}\vspace{2mm}
	\label{tab-comparacao-nfc-vs-rfid}
	\resizebox{\textwidth}{!}{%
		\begin{tabular}{m{6cm}|C{3cm}|C{3cm}}
			
			\textbf{} & \textbf{\acrshort{nfc}} & \textbf{\acrshort{rfid}} \\
			\hline Gama de frequência & 13,56MHz & 125kHz - 960MHz \\
			\hline Comunicação & Unidirecional ou bidirecional (P2P) & Unidirecional \\
			\hline Distância & até 5cm & até 100m \\
			\hline Componentes & Leitor \acrshort{nfc} e \textit{tag} \acrshort{nfc} & Leitor \acrshort{rfid}, \textit{tag} \acrshort{rfid} e uma antena \\
			\hline Suporte em \textit{smartphones} & Na maioria dos \textit{smartphones} & Não \\
			\hline Ativo/Passivo & Passivo, ativado na presença de um leitor \acrshort{nfc} & Passivo, ativado na presença de um leitor \acrshort{rfid} e Ativo, a \textit{tag} tem uma fonte de energia própria. \\
			\hline Uso aplicacional & Propriedade desenvolvidas para pagamentos móveis seguros & Usado globalmente para gestão de stocks, manipulação de bagagem no aeroporto, identificação de gado entre outros \\
			\hline Capacidade das \textit{tags} & 64 Bytes até 1024 Bytes & 96 Bits até 512 Bits ou até 4k ou 8k Bytes \\
			
		\end{tabular}
	}
\end{table}

Conforme se pode observar na tabela \ref{tab-comparacao-nfc-vs-rfid}, a comunicação com as \textit{tags} \acrshort{nfc} pode ser bidirecional o que torna mais fácil e intuitiva a transmissão entre telemóveis e \textit{tags} \acrshort{nfc}. Desta forma, optou-se pela utilização de \textit{tags} \acrshort{nfc} para os produtos avulsos em que é preciso usar o telemóvel para as programar, e, atualmente, os telemóveis só têm suporte para a tecnologia \acrshort{nfc}. Assim os locais de armazenamento têm obrigatoriamente de dispor de leitor de \textit{tags} \acrshort{nfc}. Contudo, se os embaladores decidirem utilizar \acrshort{rfid}, o hardware pode também ter leitor \acrshort{rfid}. O levantamento de requisitos feito, juntamente com a dimensão dos campos da base de dados, e sabendo que o formato utilizado na escrita das \textit{tags} é \acrlong{csv} \cite{RFC4180:csv}, estimou-se que a capacidade de armazenamento mínima das \textit{tags} é aproximadamente 324 Bytes.


\subsection{Dispositivos de \textit{Hardware}}

Posto que a componente de \textit{hardware}, sensores e leitores de \textit{tags}, não foi âmbito do projeto, mas é parte integrante e essencial para o correto funcionamento do sistema desenvolvido, foi fundamental efetuar análise e pesquisa alusiva ao assunto. Por isso, estudou-se como se deveria proceder para incorporar os dispositivos \textit{hardware} no sistema Smart Stocks. 

Assim, para adicionar um dispositivo de \textit{hardware} usar-se-ia um \textit{QRCode} \cite{qrcode:about}. Este \textit{QRCode} deveria conter informação acerca do dispositivo, como o seu identificador e possivelmente o local de armazenamento a que se destina. Este ao ser comprado e sem ter sido ainda instalado no local de armazenamento, seria adicionado à casa, pelo utilizador, recorrendo a uma funcionalidade extra da aplicação móvel. O utilizador passaria o \textit{scanner} \textit{QRCode} sobre o \textit{QRCode} presente no dispositivo de \textit{hardware} e assim a aplicação móvel comunicaria com a \gls{api-web} para associar aquele dispositivo à casa do utilizador. Caso este utilizador tenha várias casas, este teria de, previamente, especificar a qual casa pretendia associar o dispositivo. Uma vez associado, o dispositivo necessita de comunicar com a \gls{api-web} de forma a registar os movimentos presentes ocorridos naquele local de armazenamento. Para tal, quando o dispositivo for instalado e ligado, este faria um \textit{ping} à \gls{api-web} a sinalizar que está ativo e enviando a sua informação, nomeadamente o seu identificador. A \gls{api-web} respondia com o link ao qual o dispositivo \textit{hardware} teria de enviar a informação dos movimentos que deteta.

%
% Subsecção 2.3.2 Estrutura
%
\subsection{Arquitetura por Camadas}\label{subsec232}

O sistema de gestão de stocks é composto por 2 blocos principais: o bloco do lado do cliente e o bloco do lado do servidor, que se relacionam. A representação destes blocos é apresentada na Figura \ref{project-layers-structure}.

A arquitetura do projeto segue uma arquitetura por camadas, dado que este padrão permite individualizar cada camada \cite{Haque:2007:ADA:1698307.1698331}. Assim, estas tornam-se independentes umas das outras, fornecendo não só abstração sobre as camadas inferiores, mas também, oferecendo a possibilidade de testar e/ou substituir cada uma das camadas de forma independente, desde que seja mantido o contrato.

\begin{figure}[H]
	\centering
	\includegraphics[width=\textwidth, scale=1]{img/project.png}
	\caption{Arquitetura por Camadas do Projeto}
	\label{project-layers-structure}
\end{figure}

No lado do cliente existem três camadas: a camada Apresentação que é responsável por representar os dados solicitados pelo utilizador; o Controlo que está encarregue de despoletar ações na camada do \textit{Web Service} de forma a satisfazer as solicitações do utilizador; e assim o \textit{Web Service} interage com a \gls{api-web}. 

As camadas que compõem o lado do servidor são: o Controlo que processa pedidos e retorna uma resposta; a camada da Lógica de Negócio que é responsável por satisfazer as regras de negócio; e por fim o Acesso a Dados que efetua leituras e escritas sobre a \acrshort{bd}.

\subsubsection{Tecnologias Inerentes à Solução}\label{subsec233}

O lado do servidor incluí três camadas e expõe uma \gls{api-web}. A \acrfull{dal} é produzida com a linguagem de programação \textit{Java}, usando a \acrfull{jpa}, e é responsável pelas leituras e escritas sobre a \acrfull{bd}. A \acrshort{bd} é externa ao servidor, utilizando para isso o \acrfull{sgbd} \textit{PostgreSQL}. A \acrfull{bll} é responsável pela aplicação das regras de negócio. A implementação desta camada é também realizada com linguagem \textit{Java}. Os \textit{controllers} foram desenvolvidos em \textit{Java} com a \textit{framework} da \textit{Spring}, chamada de \textit{Spring Boot}\footnote{https://spring.io/projects/spring-boot}. A \gls{api-web} disponibiliza recursos em diferentes \textit{hypermedias}. Para a implementação do algoritmo de previsão de stocks usou-se a linguagem \textit{R}.

Do lado do cliente existem dois modos de interação: usando uma aplicação móvel ou usando uma aplicação web. A aplicação móvel está disponível para a plataforma \textit{Android}, e foi desenvolvida na linguagem \textit{Kotlin}. A aplicação web é compatível com a maioria dos \textit{browsers}, e é implementada utilizando a linguagem \textit{JavaScript} com o auxilio da biblioteca \textit{React}\footnote{https://reactjs.org}.