%
% Secção 2.2
%
\section{Arquitetura da Solução}\label{sec22}

Nesta secção pretende-se abordar de forma geral a solução implementada para resolver o problema apresentado na secção \ref{sec21}.

%
% Subsecção 2.2.1 Abordagem
%
\subsection{Abordagem}\label{subsec221}

\begin{figure}[H]
	\centering
	\includegraphics[width=16cm, height=8cm, scale=1]{./figures/architecture.png}
	\caption{Arquitetura Geral do Projeto}
	\label{project-general-architecture}
\end{figure}

Após uma ida às compras, os itens adquiridos, Figura \ref{project-general-architecture}(a), são armazenados nos seus respetivos locais, Figura \ref{project-general-architecture}(b). Como forma de automatizar a recolha de informação relativa quer aos artigos obtidos quer às suas caraterísticas, utilizam-se sensores. O uso destes só é possível caso os rótulos dos itens se encontrem em formato digital \textit{standard}, com \textit{tags} \acrfull{nfc} \cite{nfcforum:nfc} ou \acrfull{rfid} \cite{rfidinc:rfid}, e os locais de armazenamento contenham os respetivos leitores de \textit{tags}.

Ao guardar os artigos nos locais de armazenamento, os mesmos devem ser lidos pelos leitores, de forma a que a informação presente na \textit{tag} e o tipo de movimento (entrada ou saída) possam ser enviados para a API. Assim, estes dados são posteriormente tratados e armazenados de forma persistente na \acrfull{bd}, Figura \ref{project-general-architecture}(d). A APP, Figura \ref{project-general-architecture}(c), é responsável por retornar dados para as aplicações cliente, Figura \ref{project-general-architecture}(e, f). É ainda nesta que está presente o algoritmo de previsão de stocks utilizado para efetuar a previsão quanto à duração de cada um dos itens em stock.

No contexto da gestão de stocks assume-se a existência de duas formas de apresentação para os itens em stock: avulsos e embalados. Os primeiros são conservados em sistemas de arrumação identificados com \textit{tags} programáveis por \textit{smartphones}, \ref{project-general-architecture}(e). Os detalhes dos itens são especificados pelo utilizador e carregados para a \textit{tag}. Já os segundos contêm os seus rótulos digitais com o detalhe guardado pelos embaladores.\\

%
% Subsecção 2.2.2 Estrutura
%
\subsection{Arquitetura por Camadas}\label{subsec222}

O sistema de gestão de stocks é composto por 2 blocos principais: o bloco do lado do cliente e o bloco do lado do servidor, que se relacionam. A representação destes blocos é apresentada na Figura \ref{project-layers-structure}.

A arquitetura do projeto segue o padrão de arquitetura por camadas, dado que este padrão permite individualizar cada camada. Com isto, estas tornam-se independentes umas das outras, fornecendo não só abstração sobre as camadas inferiores, mas também, oferecendo a possibilidade de testar ou substituir cada uma das mesmas, sem que existam alterações significativas nas restantes.

\begin{figure}[H]
	\centering
	\includegraphics[width=\textwidth, scale=1]{./figures/project.png}
	\caption{Arquitetura por Camadas do Projeto}
	\label{project-layers-structure}
\end{figure}

No lado do cliente têm-se três camadas: a camada Apresentação que é responsável por representar os dados solicitados pelo utilizador; o Controlo que está encarregue de despoletar ações na camada do \textit{Web Service} de forma a satisfazer as solicitações do utilizador; e assim o \textit{Web Service} interage com a \gls{api-web}. 

As camadas que compõem o lado do servidor são: o Controlo que processa pedidos e retorna uma resposta; a camada da Lógica de Negócio que é responsável por satisfazer as regras de negócio; e por fim o Acesso a Dados que efetua leituras e escritas sobre a \acrshort{bd}.

\subsubsection{Tecnologias Inerentes à Solução}

O lado do servidor incluí três camadas e expõe uma \gls{api-web}. A \acrfull{dal} é produzida com a linguagem de programação \textit{Java}, usando a \acrfull{jpa}, e é responsável pelas leituras e escritas exercidas sobre a \acrfull{bd}. A \acrshort{bd} é externa ao servidor, utilizando para isso o \acrfull{sgbd} \textit{PostgreSQL}. A \acrfull{bll} é responsável pela aplicação das regras de negócio. A implementação desta camada é também realizada com linguagem \textit{Java}. Os \textit{controllers} foram desenvolvidos em \textit{Java} com a \textit{framework} da \textit{Spring}, chamada de \textit{Spring Boot}. A \gls{api-web} disponibiliza recursos em diferentes \textit{hypermedias}. Para a implementação do algoritmo de previsão de stocks usou-se a linguagem \textit{R}.

Do lado do cliente existem dois modos de interação, por uma aplicação móvel e outra por uma aplicação web. A aplicação móvel disponível para a plataforma \textit{Android}, desenvolvida em linguagem \textit{Kotlin}. A aplicação web é disponibilizada para a maioria dos browsers, e é implementada utilizando a linguagem \textit{JavaScript} no ambiente \textit{Node.js}, com o auxilio da \textit{framework Express}.