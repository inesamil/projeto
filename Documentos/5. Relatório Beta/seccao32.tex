%
% Base de Dados
%
\section{Base de Dados}\label{sec32}

Os dados são armazenados de forma persistente numa \acrfull{bd}. A \acrshort{bd} implementada é relacional uma vez que não se preveem alterações durante o uso, ou seja, as tabelas são de certa forma estáticas, não necessitando portanto do dinamismo oferecido por uma \acrshort{bd} documental, por exemplo. 

 A escolha de qual o melhor \acrfull{sgbd} assentava em três possibilidades, \textit{SQL Server}, \textit{PostgreSQL} e \textit{MySQL}. O primeiro apesar de ser uma ferramenta com a qual o grupo estava familiarizado foi automaticamente excluída visto que um dos requisitos exigidos era ser \gls{open-source}, caraterística não presente nesta ferramenta. De seguida, ambas as ferramentas são \gls{open-source} e têm uma elevada compatibilidade com os principais fornecedores de serviços \textit{cloud}. Pelo que a verdadeira distinção se prende com os factos:
 	\begin{itemize}
 		\item O \textit{PostgreSQL} é compatível com as propriedades \acrfull{acid}, garantindo assim que todos os requisitos sejam atendidos;
 		\item O \textit{PostgreSQL} aborda a concorrência de uma forma eficiente com a sua implementação de \acrfull{mvcc}, que alcança níveis muito altos de concorrência;
 		\item O \textit{PostgreSQL} possui vários recursos dedicados à extensibilidade. É possível adicionar novos tipos, novas funções, novos tipos de índice, etc.
	\end{itemize}
 Assim sendo, foi escolhido o \acrfull{sgbdro} \textit{PostgreSQL}, como já anteriormente mencionado, na secção \ref{sec13} do capítulo \ref{cap1}.
 
 \subsection{Implementação}\label{subsec321}
 
Na \acrshort{bd} foram desenvolvidas funções que garantem a consistência dos dados, por um lado na inserção de entidades cujos \textit{IDs} sejam incrementais ou gerados consoante o desejado, por outro lado na remoção  de entidades que se relacionam com outras.

Decidiu-se usar funções na \acrshort{bd} em vez de criar métodos em \textit{Java}, pois se imaginarmos um cenário onde a aplicação servidora esteja distribuída, existe um problema no controlo da concorrência na geração dos \textit{ID}s. Tendo em conta que a \acrshort{bd} não distribuída não existe o problema descrito.
