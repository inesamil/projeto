% Página de resumo em Português
\cleardoublepage\newpage
\chapter*{Resumo} \label{resumo}

A gestão de stocks é uma tarefa estruturada e repetitiva. Como tal, por meio de uma aplicação \textit{mobile} e \textit{web} com suporte inteligente de um algoritmo de previsão de stocks pretende-se solucionar este problema.

Tendo por base a automatização da recolha de dados simplifica-se, não só, o controlo de stocks, como também, a análise dos padrões de consumo e reposição.
Desta forma, auxilia-se os utilizadores a manter o stock adequado às suas necessidades, bem como alertá-los para a proximidade do fim da validade e/ou stock dos produtos. 

Assim, este trabalho vai no sentido de responder a questões como: ``De que forma podemos evitar transtornos causados na altura de reabastecer a nossa despensa? Ou como proceder ao controlo de stocks de alimentos e outros produtos? E como impedir artigos fora de prazo?". Se se entender que uma casa funciona como uma empresa e existem quantidades mínimas recomendadas, é possível gerar uma nota de encomenda com os produtos em falta ou prestes a terminar para o utilizador consultar e exercer a compra.

\vspace{0.2cm}
{\bf Palavras-chave:} análise; automatização; gestão; previsão; stocks; tarefas.