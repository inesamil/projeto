%
% Capítulo 2
%
\chapter{Gestão de Stocks} \label{cap2}

Neste capítulo a aplicação Smart Stocks é descrita na secção \ref{sec21}, bem como os requisitos funcionais e opcionais na secção \ref{sec22}.


%
% Secção 2.1
%
\section{Aplicação Smart Stocks} \label{sec21}
A Smart Stocks é uma aplicação que visa dar suporte à gestão de stocks domésticos. Com fim de se alcançar o desejado é necessário recolher determinadas informações, tais como, as características da casa a gerir, as particularidades dos membros co-habitantes da casa e ainda os padrões de consumo e reposição. Para facilitar tal tarefa são disponibilizadas listas geridas pelo sistema. Por exemplo, lista de compras e lista dos itens em stock na casa, cuja consistência é garantida às custas dos movimentos de entrada e saída dos itens nos diversos locais de armazenamento.

Em seguida listam-se as diversas entidades que compõem o sistema de informação que permita gerir os itens em stock numa dada casa.
\subsubsection{Casa}
\begin{itemize}
	\item Cada casa é caracterizada por um identificador único, um nome, atribuído por um utilizador no momento de registo da casa. O número de bebés, crianças, adultos e seniores que vivem nessa casa.
	\item Uma casa está associada a um ou mais utilizadores, podendo um utilizador ter várias casas. 
	\item Podem existir um ou mais administradores.
	\item A casa pode ter vários itens em stock presentes na mesma.
	\item Para cada casa existem vários locais de armazenamento dos itens, por exemplo armários, frigoríficos, etc.
	\item Em cada casa deve ser possível conhecer as alergias assim como quantos membros possuem essa alergia (os membros não precisam necessariamente de estar registados).
\end{itemize}

\subsubsection{Utilizador}
\begin{itemize}
	\item Uma pessoa é representada por um utilizador que é identificado univocamente por um email ou por um nome de utilizador, pelo nome da pessoa, a sua idade e uma password.
\end{itemize}

\subsubsection{Listas}
\begin{itemize}
	\item Cada lista é composta por um identificador único e um nome.
	\item Uma lista pode ter vários produtos.
	\item Existem dois tipos de listas: de sistema e de utilizador. 
	\item As listas de sistema são comuns a todos os utilizadores registados, contudo são particulares a cada casa. 
	\item Um utilizador pode criar as suas listas, partilhando-as com outros utilizadores da casa ou tornar a lista secreta.
\end{itemize}

\subsubsection{Categoria}
\begin{itemize}
	\item Uma categoria é identificada univocamente por um número ou por um nome.
\end{itemize}


\subsubsection{Produtos}
\begin{itemize}
	\item Um produto é constituído por um identificador único, um nome, se é ou não comestível, e a validade perecível.
	\item Para os produtos presentes numa lista pode ser possível saber a sua marca e a quantidade.
	\item Um produto pertence a uma categoria, podendo uma categoria ter vários produtos.
	\item Um produto pode ser concretizado por diversos itens em stock na casa.
\end{itemize}
 
\subsubsection{Item em Stock}
\begin{itemize}
	\item Um item em stock é a concretização de um produto que existe numa casa. É identificado univocamente por um número ou por uma marca, uma variedade e um segmento, é também caracterizado por uma descrição, o local de conservação, a quantidade e as datas de validade. 
	\item Para cada item deve ser possível saber os seus movimentos de entrada e saída de um local de armazenamento.
	\item Deve também ser possível saber os alergénios de um item presente na casa.
\end{itemize}

\subsubsection{Movimento}
\begin{itemize}
	\item Para cada movimento deve ser possível saber o tipo do movimento (entrada ou saída), a data em que ocorreu e a quantidade de produtos. 
\end{itemize}

\subsubsection{Local de armazenamento}
\begin{itemize}
	\item Cada local de armazenamento é caraterizado por um identificador único, a temperatura e um nome.
	\item Deve ser possível saber a quantidade de cada item presente no local.
	\item Um local de armazenamento pode ter vários itens em stock presentes na casa e estar associado a diversos movimentos.
\end{itemize}


%
% Secção 2.2
%
\section{Requisitos Funcionais e Opcionais} \label{sec22}

\paragraph{Requisitos Funcionais}
\begin{itemize}
	\item Informar o utilizador dos produtos existentes, a sua validade e a sua quantidade;
	\item Alertas sobre os produtos que estão perto da data de validade;
	\item Geração da lista de compras com os produtos em falta;
	\item Possibilidade de especificar os produtos a ter sempre em stock bem como as suas quantidades mínimas;	
	\item Lista de Compras \textit{Offline};
	\item Listas partilhadas entre utilizadores da mesma casa;
	\item Criação de Listas;
	\item Especificação das alergias dos membros da casa.
\end{itemize}


\paragraph{Requisitos Opcionais}
\begin{itemize}
	\item Lista de produtos quase a expirar;
	\item Lista de produtos indesejados (Lista Negra);
	\item Lista de contenção em situações de emergência (Lista SOS);
	\item Inserir refeições extraordinárias de eventos a realizar num futuro próximo, para acrescentar alimentos não básicos à lista de compras;
\end{itemize}


%
% Secção 2.3
%
\section{Dificuldades Encontradas} \label{sec23}

Para a realização deste projeto encontrámos dificuldades nos aspetos a seguir referidos.

\subsection{Rótulos em Formato Não Digital}

Nos dias de hoje, os produtos não possuem rótulos digitais. Isto é um problema para a concretização do projeto, na medida em que se torna menos eficiente a recolha dos dados presentes nos produtos. Contudo, assumindo que este problema é resolvido fora do âmbito do projeto, apenas é preciso definir um formato standard de como os dados devem ser armazenados nas \textit{tags}, que podem ser \acrshort{nfc} ou \acrshort{rfid}. Num cenário ideal, este formato deve ser respeitado por todos os embaladores. Assim, os produtos têm um rótulo, código de barras e uma \textit{tag} \acrshort{nfc} ou \acrshort{rfid}, com a informação necessária. Está fora do âmbito do trabalho implementar o suporte hardware para a leitura das \textit{tags} e qual o sentido do movimento (entrada ou saída). Assume-se que essas informações são disponibilizadas num formato conhecido. 


\subsection{Ausência de Identificador Único nos Itens}

Os itens não dispõem de um identificador unívoco. Alguns deles contêm um lote e um número de série. A ausência deste identificador impede a distinção entre itens iguais, o que impossibilita saber se entrou um novo item no local de armazenamento ou se saiu um dos itens presentes. Tal facto torna a gestão dos stocks dependente do dispositivo de hardware para distinguir o tipo de movimento.